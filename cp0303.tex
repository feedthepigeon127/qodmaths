\documentclass[varwidth=70mm]{standalone}
%standard width: 70 mm
\usepackage[UKenglish]{babel}

\usepackage[utf8]{inputenc}
\usepackage[T1]{fontenc}
\usepackage{libertine}
\usepackage[libertine]{newtxmath}

\usepackage[autostyle]{csquotes}
\usepackage[free-standing-units, binary-units]{siunitx}
\sisetup{%
locale=UK,
range-phrase=--,
table-figures-decimal=2,
table-auto-round=true,
}
\DeclareSIUnit\Molar{\textsc{m}}

\usepackage{chemmacros}
\usepackage{amsmath, amsfonts, amssymb}
\usepackage{textcomp}
\usepackage{bm}
\usepackage{xfrac}

\usepackage{varwidth}
\usepackage{comment}
\makeatletter
\g@addto@macro{\@parboxrestore}{\parskip=\baselineskip}
\makeatother

\usepackage[inline]{asymptote}

\newcounter{choice}
\renewcommand\thechoice{\Alph{choice}}
\newcommand\choicelabel{\thechoice.}

\newenvironment{choices}%
  {\list{\choicelabel}%
     {\usecounter{choice}\def\makelabel##1{\hss\llap{##1}}%
       \settowidth{\leftmargin}{W.\hskip\labelsep\hskip 2.5em}%
       \def\choice{%
         \item
       } % choice
       \labelwidth\leftmargin\advance\labelwidth-\labelsep
       \topsep=0pt
       \partopsep=0pt
     }%
  }%
  {\endlist}

\newenvironment{oneparchoices}%
  {%
    \setcounter{choice}{0}%
    \def\choice{%
      \refstepcounter{choice}%
      \ifnum\value{choice}>1\relax
        \penalty -50\hskip 1em plus 1em\relax
      \fi
      \choicelabel
      \nobreak\enskip
    }% choice
    % If we're continuing the paragraph containing the question,
    % then leave a bit of space before the first choice:
    \ifvmode\else\enskip\fi
    \ignorespaces
  }%
  {}
\excludecomment{answer}
\includecomment{answer}
          
\begin{document}
\begin{center}
\begin{asy}
size(50mm);
import geometry;

real rdeg = 10; 
real r = 10;
real cdr = 3/4;
real alpha = aCos(cdr/2);
real lde = sqrt(4*r^2 + (cdr*r)^4/r^2 - 4*(cdr*r)^2);

point C = rotate(rdeg)*(-r, 0); point D = rotate(rdeg)*(r, 0);
point A = rotate(alpha+rdeg, C)*(C + cdr*r);
point B = rotate(-alpha+rdeg, C)*(C + cdr*r);
point E = rotate(2*alpha+rdeg, D)*(D + lde);

circle cle = circle((point)(0, 0), r);
segment CD = segment(C, D); segment AB = segment(A, B);
segment CA = segment(C, A); segment AE = segment(A, E); segment ED = segment(E, D);

draw(cle);
draw(CD);
draw(AB, StickIntervalMarker(i=2, n=1, black, size=1mm));
draw(CA, StickIntervalMarker(i=1, n=2, black, size=1mm));
draw(AE, StickIntervalMarker(i=1, n=2, black, size=1mm));
draw(CA); draw(AE); draw(ED);

dot("$C$", C, W);
dot("$D$", D);
dot("$A$", A, NW);
dot("$B$", B, SW);
dot("$E$", E, N);

\end{asy}
\end{center}

In the figure, $A$, $B$, $C$, $D$ and $E$ are points on the circle with radius $r$. $AB$ is a chord of the circle and $CD$ is a perpendicular bisector of $AB$. $ACDE$ is a quadrilateral with $AC=AE=\frac{3}{4}r$. What is the length of the chord $DE$ in term of $r$?

\begin{choices}
\choice $\dfrac{11}{8}r$
\choice $\dfrac{23}{16}r$%
\choice $\dfrac{47}{32}r$
\choice $\dfrac{7}{4}r$
\end{choices}

\begin{answer}
\hrulefill\par
\textbf{Ans: B}

\begin{center}
\begin{asy}
size(50mm);
import geometry;

real rdeg = 10; 
real r = 10;
real cdr = 3/4;
real alpha = aCos(cdr/2);
real lde = sqrt(4*r^2 + (cdr*r)^4/r^2 - 4*(cdr*r)^2);

point C = rotate(rdeg)*(-r, 0); point D = rotate(rdeg)*(r, 0);
point A = rotate(alpha+rdeg, C)*(C + cdr*r);
point B = rotate(-alpha+rdeg, C)*(C + cdr*r);
point E = rotate(2*alpha+rdeg, D)*(D + lde);

circle cle = circle((point)(0, 0), r);
segment CD = segment(C, D); segment AB = segment(A, B);
segment CA = segment(C, A); segment AE = segment(A, E); segment ED = segment(E, D);

draw(cle);
draw(CD);
draw(AB, StickIntervalMarker(i=2, n=1, black, size=1mm));
draw(CA, StickIntervalMarker(i=1, n=2, black, size=1mm));
draw(AE, StickIntervalMarker(i=1, n=2, black, size=1mm));
draw(CA); draw(AE); draw(ED);
draw(A--D, dashed); draw(C--E, dashed);

dot("$C$", C, W);
dot("$D$", D);
dot("$A$", A, NW);
dot("$B$", B, SW);
dot("$E$", E, N);
perpendicularmark(segment(A,D), CA, size=5, quarter=4);
perpendicularmark(segment(C,E), ED, size=5, quarter=3);

\end{asy}
\end{center}

Consider $\bigtriangleup CDE$, by Pythagorean theorem,
\begin{equation*}
\begin{aligned}
DE^2 + CE^2 &= CD^2
\end{aligned}
\end{equation*}
As $CD$ is a perpendicular bisector of a chord, it must pass through the centre of the circle. Since $C$ and $D$ are also points on the circle, $CD$ is the diameter of the circle and $CD=2r$. Note that $CE=AB$, as they have the same subtended angle at the centre.
\begin{equation*}
\begin{aligned}
\therefore DE^2 &= CD^2 - CE^2 \\
     &= (2r)^2 - AB^2
\end{aligned}
\end{equation*}
Consider the area of $\bigtriangleup ACD$,
\begin{equation*}
\begin{aligned}
\frac{1}{2}\times AC\times AD &= \frac{1}{2}\times CD\times\frac{AB}{2} \\
\frac{3}{4}r\cdot AD &= 2r\cdot \frac{AB}{2} \\
AB &= \frac{3}{4}AD
\end{aligned}
\end{equation*}
Consider $\bigtriangleup ACD$, by Pythagorean theorem,
\begin{equation*}
\begin{aligned}
AD^2 + AC^2 &= CD^2 \\
AD^2 &= CD^2 - AC^2 \\
AD^2 &= 4r^2 - \frac{9}{16}r^2 \\
	 &= \frac{55}{16}r^2
\end{aligned}
\end{equation*}
\begin{equation*}
\begin{aligned}
\therefore DE^2 &= 4r^2 - AB^2 \\
     &= 4r^2 - \frac{9}{16}AD^2 \\
     &= 4r^2 - \frac{9}{16}\cdot\frac{55}{16}r^2 \\
     &= \frac{529}{256}r^2 \\
\end{aligned}
\end{equation*}
$$\therefore\quad DE = \frac{23}{16}r$$
\end{answer}
\end{document}