\documentclass[border=3pt,varwidth=70mm]{standalone}
%standard width: 70 mm
\usepackage[UKenglish]{babel}

\usepackage[utf8]{inputenc}
\usepackage[T1]{fontenc}
\usepackage{libertine}
\usepackage[libertine]{newtxmath}

\usepackage[autostyle]{csquotes}
\usepackage[free-standing-units, binary-units]{siunitx}
\sisetup{%
locale=UK,
range-phrase=--,
table-figures-decimal=2,
table-auto-round=true,
}
\DeclareSIUnit\Molar{\textsc{m}}

\usepackage{chemmacros}
\usepackage{amsmath, amsfonts, amssymb}
\usepackage{textcomp}
\usepackage{bm}
\usepackage{xfrac}

\usepackage{varwidth}
\usepackage{comment}
\makeatletter
\g@addto@macro{\@parboxrestore}{\parskip=\baselineskip}
\makeatother

\usepackage[inline]{asymptote}

\newcounter{choice}
\renewcommand\thechoice{\Alph{choice}}
\newcommand\choicelabel{\thechoice.}

\newenvironment{choices}%
  {\list{\choicelabel}%
     {\usecounter{choice}\def\makelabel##1{\hss\llap{##1}}%
       \settowidth{\leftmargin}{W.\hskip\labelsep\hskip 2.5em}%
       \def\choice{%
         \item
       } % choice
       \labelwidth\leftmargin\advance\labelwidth-\labelsep
       \topsep=0pt
       \partopsep=0pt
     }%
  }%
  {\endlist}

\newenvironment{oneparchoices}%
  {%
    \setcounter{choice}{0}%
    \def\choice{%
      \refstepcounter{choice}%
      \ifnum\value{choice}>1\relax
        \penalty -50\hskip 1em plus 1em\relax
      \fi
      \choicelabel
      \nobreak\enskip
    }% choice
    % If we're continuing the paragraph containing the question,
    % then leave a bit of space before the first choice:
    \ifvmode\else\enskip\fi
    \ignorespaces
  }%
  {}
\excludecomment{answer}
%\includecomment{answer}
           
\begin{document}
\begin{center}
\begin{asy}
size(65mm);
import geometry;
import myasy;

real r = 1; real theta = 50; real alpha = (180-3*theta)/2;
real vang = 110; real ofs = vang-90;
circle cle = circle(origin, r);

point A = angpoint(cle, vang);
point B = intersectionpoints(cle, line(A,A+dir(180+theta+ofs)))[0];
point C = intersectionpoints(cle, line(A,A+dir(-theta+ofs)))[0];

point D = intersectionpoints(cle, rotate(-3*theta, C)*line(A,C))[0];
point E = intersectionpoint(line(A,C), rotate(180-alpha,D)*line(C,D));

draw(cle);
draw(A--B--C--cycle);
draw(C--D--E--cycle);

label("$A$", A, labelat(A, new pair[] {B,C}));
label("$B$", B, labelat(B, new pair[] {A,C}));
label("$C$", C, labelat(C, new pair[] {B,D}));
label("$D$", D, labelat(D, new pair[] {C,E}));
label("$E$", E, labelat(E, new pair[] {C,D}));

markangle("", C, B, A, radius=10);

write(sharpdegrees(line(C,D), line(D,E)));
write(sharpdegrees(line(C,E), line(E,D)));

\end{asy}
\end{center}

In the figure, the point $A$, $B$, $C$ and $D$ are lying on a circle such that the chord $AB=AC$ and $CD=CB$. $E$ is a point outside the circle such that $ACE$ is a straight line and $CD=CE$. If $\angle CED=\ang{15}$, then $\angle ABC$ is

\begin{choices}
\choice \ang{50}.%
\choice \ang{52}.
\choice \ang{55}.
\choice \ang{58}.
\end{choices}


\begin{answer}
\hrulefill\par
\textbf{Ans: A}
\begin{center}
\begin{asy}
size(65mm);
import geometry;
import myasy;
usepackage("siunitx");

real r = 1; real theta = 50; real alpha = (180-3*theta)/2;
real vang = 110; real ofs = vang-90;
circle cle = circle(origin, r);

point A = angpoint(cle, vang);
point B = intersectionpoints(cle, line(A,A+dir(180+theta+ofs)))[0];
point C = intersectionpoints(cle, line(A,A+dir(-theta+ofs)))[0];

point D = intersectionpoints(cle, rotate(-3*theta, C)*line(A,C))[0];
point E = intersectionpoint(line(A,C), rotate(180-alpha,D)*line(C,D));

draw(cle);
draw(A--B--C--cycle);
draw(C--D--E--cycle);
draw(B--D, dashed);
draw(A--B, StickIntervalMarker(1,1,size=2mm));
draw(A--C, StickIntervalMarker(1,1,size=2mm));
draw(B--C, StickIntervalMarker(1,2,size=2mm));
draw(D--C, StickIntervalMarker(1,2,size=2mm));
draw(C--E, StickIntervalMarker(1,2,size=2mm));

label("$A$", A, labelat(A, new pair[] {B,C}));
label("$B$", B, labelat(B, new pair[] {A,C}));
label("$C$", C, labelat(C, new pair[] {B,D}));
label("$D$", D, labelat(D, new pair[] {C,E}));
label("$E$", E, labelat(E, new pair[] {C,D}));

markangle("$\theta$", C, B, A, radius=10);
markangle("$\ang{15}$", C, E, D, radius=30);

write(sharpdegrees(line(C,D), line(D,E)));
write(sharpdegrees(line(C,E), line(E,D)));

\end{asy}
\end{center}

Let $\angle ABC$ be $\theta$. Since $AB=AC$, $\angle ACB=\theta$ and $\angle BCE=\ang{180}-\theta$.

$\angle BAC = \ang{180}-2\theta$, and since the same arc ($BC$ here) in a circle subtends same angle in the circumference, $\angle CDB = \angle BAC$. Since $CD=CB$, $\angle CBD = \angle CDB = \ang{180}-2\theta$. So $\angle BCD = \ang{180}-2(\ang{180}-2\theta) = 4\theta - \ang{180}$.

\begin{equation*}
\begin{aligned}
\because \angle DCE &= \angle BCD + \angle BCE \\
\therefore \ang{180}-2\times\ang{15} &=  4\theta - \ang{180} + \ang{180}-\theta \\
3\theta &= \ang{150} \\
\theta &= \ang{50} \\
\end{aligned}
\end{equation*}



\end{answer}
\end{document}