\documentclass[border=3pt,varwidth=70mm]{standalone}
%standard width: 70 mm
\usepackage[UKenglish]{babel}

\usepackage[utf8]{inputenc}
\usepackage[T1]{fontenc}
\usepackage{libertine}
\usepackage[libertine]{newtxmath}

\usepackage[autostyle]{csquotes}
\usepackage[free-standing-units, binary-units]{siunitx}
\sisetup{%
locale=UK,
range-phrase=--,
table-figures-decimal=2,
table-auto-round=true,
}
\DeclareSIUnit\Molar{\textsc{m}}

\usepackage{chemmacros}
\usepackage{amsmath, amsfonts, amssymb}
\usepackage{textcomp}
\usepackage{bm}
\usepackage{xfrac}

\usepackage{varwidth}
\usepackage{comment}
\makeatletter
\g@addto@macro{\@parboxrestore}{\parskip=\baselineskip}
\makeatother

\usepackage[inline]{asymptote}

\newcounter{choice}
\renewcommand\thechoice{\Alph{choice}}
\newcommand\choicelabel{\thechoice.}

\newenvironment{choices}%
  {\list{\choicelabel}%
     {\usecounter{choice}\def\makelabel##1{\hss\llap{##1}}%
       \settowidth{\leftmargin}{W.\hskip\labelsep\hskip 2.5em}%
       \def\choice{%
         \item
       } % choice
       \labelwidth\leftmargin\advance\labelwidth-\labelsep
       \topsep=0pt
       \partopsep=0pt
     }%
  }%
  {\endlist}

\newenvironment{oneparchoices}%
  {%
    \setcounter{choice}{0}%
    \def\choice{%
      \refstepcounter{choice}%
      \ifnum\value{choice}>1\relax
        \penalty -50\hskip 1em plus 1em\relax
      \fi
      \choicelabel
      \nobreak\enskip
    }% choice
    % If we're continuing the paragraph containing the question,
    % then leave a bit of space before the first choice:
    \ifvmode\else\enskip\fi
    \ignorespaces
  }%
  {}
\excludecomment{answer}
%\includecomment{answer}
           
\begin{document}
\begin{center}
\begin{asy}
size(65mm);
import geometry;
import myasy;

real r = 1; real t = 30; real tdeg = 68;
real d = r / (2*Cos(tdeg));

circle cle = circle(origin, r);
point A = rotate(t)*rotate(tdeg)*(r,0);
point B = rotate(t)*(d, 0);

point T = rotate(t)*rotate(-90)*(r, 0);
point H1 = T+unit(B-origin);
point C = intersectionpoint(line(T, H1), line(A, B));
point H2 = rotate(140,T)*(T+unit(C-T));
point[] K = intersectionpoints(line(T,H2), cle);

draw(cle);
draw(origin--A--B--cycle);
draw(T--C);
draw(B--C);
draw(T--K[1]--A);

label("$O$", origin, labelat(origin, new pair[] {A,B}));
label("$A$", A, labelat(A, new pair[] {K[1],origin,C}));
label("$B$", B, labelat(B, new pair[] {origin}));
label("$T$", T, labelat(T, new pair[] {C,K[1]}));
label("$C$", C, labelat(C, new pair[] {A,T}));
label("$K$", K[1], labelat(K[1], new pair[] {A,T}));

//label("$O$", origin, (unit(origin-A)+unit(origin-B))/2);
//label("$A$", A, (unit(A-K[1])+unit(A-origin)+unit(A-C))/3);
//label("$B$", B, unit(B-origin));
//label("$T$", T, (unit(T-C)+unit(T-K[1]))/2);
//label("$C$", C, (unit(C-A)+unit(C-T))/2);
//label("$K$", K[1], (unit(K[1]-A)+unit(K[1]-T))/2);

markangle("", T, K[1], A, radius=10);

write(sharpdegrees(line(T,K[1]), line(A,K[1])));
write(sharpdegrees(line(A,C), line(T,C)));

\end{asy}
\end{center}

In the figure, a circle is centred at $O$ with radius $OA$. $C$ is a point outside the circle such that $CT$ is a tangent to the circle and $CA$ is a straight line. $B$ is a point on $CA$ such that $BO\parallel CT$ and $BO=BA$. If $\angle ACT=\ang{44}$, then $\angle AKT$ is

\begin{choices}
\choice \ang{67}.
\choice \ang{68}.
\choice \ang{76}.
\choice \ang{79}.%
\end{choices}


\begin{answer}
\hrulefill\par
\textbf{Ans: D}

Since $BO\parallel CT$, $\angle ABO=\angle ACT=\ang{44}$. Since $BA=BO$, $\angle BOA = \angle BAO$, consider $\bigtriangleup ABO$,
\begin{equation*}
\begin{aligned}
2\angle BOA + \angle ABO &= \ang{180} \\
\angle BOA &= \frac{\ang{180}-\ang{44}}{2} \\
\angle BOA &= \ang{68} \\
\end{aligned}
\end{equation*}

Since $CT$ is a tangent to the circle, $\angle BOT=\ang{90}$ ($BO\parallel CT$).

$\angle AOT=\angle BOA + \angle BOT = \ang{68}+\ang{90} = \ang{158}$.

The angle subtended at the circumference is half the angle subtended at the centre by the same arc, so $\angle AKT = \angle AOT / 2 = \ang{158}/2 = \ang{79}$. 

\end{answer}
\end{document}