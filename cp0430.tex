\documentclass[border=3pt,varwidth=70mm]{standalone}
%standard width: 70 mm
\usepackage[UKenglish]{babel}

\usepackage[utf8]{inputenc}
\usepackage[T1]{fontenc}
\usepackage{libertine}
\usepackage[libertine]{newtxmath}

\usepackage[autostyle]{csquotes}
\usepackage[free-standing-units, binary-units]{siunitx}
\sisetup{%
locale=UK,
range-phrase=--,
table-figures-decimal=2,
table-auto-round=true,
}
\DeclareSIUnit\Molar{\textsc{m}}

\usepackage{chemmacros}
\usepackage{amsmath, amsfonts, amssymb}
\usepackage{textcomp}
\usepackage{bm}
\usepackage{xfrac}

\usepackage{varwidth}
\usepackage{comment}
\makeatletter
\g@addto@macro{\@parboxrestore}{\parskip=\baselineskip}
\makeatother

\usepackage[inline]{asymptote}

\newcounter{choice}
\renewcommand\thechoice{\Alph{choice}}
\newcommand\choicelabel{\thechoice.}

\newenvironment{choices}%
  {\list{\choicelabel}%
     {\usecounter{choice}\def\makelabel##1{\hss\llap{##1}}%
       \settowidth{\leftmargin}{W.\hskip\labelsep\hskip 2.5em}%
       \def\choice{%
         \item
       } % choice
       \labelwidth\leftmargin\advance\labelwidth-\labelsep
       \topsep=0pt
       \partopsep=0pt
     }%
  }%
  {\endlist}

\newenvironment{oneparchoices}%
  {%
    \setcounter{choice}{0}%
    \def\choice{%
      \refstepcounter{choice}%
      \ifnum\value{choice}>1\relax
        \penalty -50\hskip 1em plus 1em\relax
      \fi
      \choicelabel
      \nobreak\enskip
    }% choice
    % If we're continuing the paragraph containing the question,
    % then leave a bit of space before the first choice:
    \ifvmode\else\enskip\fi
    \ignorespaces
  }%
  {}
%\excludecomment{answer}
\includecomment{answer}
           
\begin{document}
\begin{center}
\begin{asy}
size(65mm);
import geometry;

real p = 10; real q = 15; real ABF = 40;

point B = (0, 0);
point C = B + (p+q, 0);
point F = B + (p, 0);
point A = B + (p/2, 2*ABF/p);
point D = A + (p+q, 0);
point G = A + (p, 0);
point E = B + (p/2, -2*ABF/p);
point X = intersectionpoint(segment(A,C), segment(F,G));


draw(A--B--C--D--cycle);
draw(A--F);
draw(A--C);
draw(E--B);
draw(E--F);
draw(F--G);

label("$A$", A, (unit(A-B)+unit(A-F)+unit(A-C)+unit(A-D))/4);
label("$B$", B, (unit(B-A)+unit(B-C)+unit(B-E))/3);
label("$C$", C, (unit(C-A)+unit(C-B)+unit(C-D))/3);
label("$D$", D, (unit(D-A)+unit(D-C))/2);
label("$E$", E, (unit(E-B)+unit(E-F))/2);
label("$F$", F, unit(F-A));
label("$G$", G, unit(G-X));
label("$X$", X, (unit(X-A)+unit(X-G))/2);


\end{asy}
\end{center}

In the figure, $ABCD$ is a parallelogram and $ABEF$ is a rhombus. $EF$ produced meets $AD$ at the point $G$. $AC$ and $FG$ intersect at the point $X$. If $BF:FC=2:3$ and the area of the rhombus $ABEF$ is \SI{80}{\centi\meter\squared}, then the area of the quadrilateral $CDGX$ is 

\begin{choices}
\choice \SI{76}{\centi\meter\squared}.
\choice \SI{84}{\centi\meter\squared}.%
\choice \SI{92}{\centi\meter\squared}.
\choice \SI{100}{\centi\meter\squared}.
\end{choices}


\begin{answer}
\hrulefill\par
\textbf{Ans: B}

$ABFG$ is a parallelogram because $AB\parallel EF\parallel FG$ and $AG\parallel BF$, so $BF=AG$.

$AB=AF=EB=EF$ because $ABEF$ is a rhombus, so $\bigtriangleup ABF\cong\bigtriangleup FGA$. Therefore, area of $\bigtriangleup ABF = \bigtriangleup EBF = \bigtriangleup FGA = \SI{40}{\centi\meter\squared}$. 

$\because\bigtriangleup AGX\sim\bigtriangleup CFX$,
\begin{equation*}
\begin{aligned}
\frac{GX}{FX} &= \frac{AG}{CF} \\
\therefore \frac{GX}{FX} &= \frac{2}{3} \\ 
\end{aligned}
\end{equation*}

$\because\bigtriangleup AFX$ and $\bigtriangleup AGX$ share the same height,
\begin{equation*}
\begin{aligned}
\frac{\bigtriangleup AGX}{\bigtriangleup AFX} &= \frac{GX}{FX} \\
\therefore\frac{\bigtriangleup AGX}{\bigtriangleup AFX} &= \frac{2}{3} \\
\end{aligned}
\end{equation*}
\begin{equation*}
\begin{aligned}
\because\bigtriangleup AFX + \bigtriangleup AGX &= \bigtriangleup FGA\\
\therefore\frac{3}{2}\bigtriangleup AGX + \bigtriangleup AGX &= \SI{40}{\centi\meter\squared} \\
\bigtriangleup AGX &= \SI{16}{\centi\meter\squared} \\
\end{aligned}
\end{equation*}

$\because\bigtriangleup ABF$ and $\bigtriangleup ABC$ share the same height,
\begin{equation*}
\begin{aligned}
\frac{\bigtriangleup ABF}{\bigtriangleup ABC} &= \frac{BF}{BC} \\
\frac{\bigtriangleup ABF}{\bigtriangleup ABC} &= \frac{2}{5} \\
\therefore\bigtriangleup ABC &= \frac{5}{2}\bigtriangleup ABF \\
							&= \SI{100}{\centi\meter\squared} \\
\end{aligned}
\end{equation*}

\begin{equation*}
\begin{aligned}
\therefore CDGX &= \bigtriangleup ACD - \bigtriangleup AGX \\ 
				 &= \bigtriangleup ABC - \bigtriangleup AGX \\
				 &= 100 - 16 \\
				 &= \SI{84}{\centi\meter\squared} \\
\end{aligned}
\end{equation*}

\end{answer}
\end{document}