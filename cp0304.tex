\documentclass[varwidth=70mm]{standalone}
%standard width: 70 mm
\usepackage[UKenglish]{babel}

\usepackage[utf8]{inputenc}
\usepackage[T1]{fontenc}
\usepackage{libertine}
\usepackage[libertine]{newtxmath}

\usepackage[autostyle]{csquotes}
\usepackage[free-standing-units, binary-units]{siunitx}
\sisetup{%
locale=UK,
range-phrase=--,
table-figures-decimal=2,
table-auto-round=true,
}
\DeclareSIUnit\Molar{\textsc{m}}

\usepackage{chemmacros}
\usepackage{amsmath, amsfonts, amssymb}
\usepackage{textcomp}
\usepackage{bm}
\usepackage{xfrac}

\usepackage{varwidth}
\usepackage{comment}
\makeatletter
\g@addto@macro{\@parboxrestore}{\parskip=\baselineskip}
\makeatother

\usepackage[inline]{asymptote}

\newcounter{choice}
\renewcommand\thechoice{\Alph{choice}}
\newcommand\choicelabel{\thechoice.}

\newenvironment{choices}%
  {\list{\choicelabel}%
     {\usecounter{choice}\def\makelabel##1{\hss\llap{##1}}%
       \settowidth{\leftmargin}{W.\hskip\labelsep\hskip 2.5em}%
       \def\choice{%
         \item
       } % choice
       \labelwidth\leftmargin\advance\labelwidth-\labelsep
       \topsep=0pt
       \partopsep=0pt
     }%
  }%
  {\endlist}

\newenvironment{oneparchoices}%
  {%
    \setcounter{choice}{0}%
    \def\choice{%
      \refstepcounter{choice}%
      \ifnum\value{choice}>1\relax
        \penalty -50\hskip 1em plus 1em\relax
      \fi
      \choicelabel
      \nobreak\enskip
    }% choice
    % If we're continuing the paragraph containing the question,
    % then leave a bit of space before the first choice:
    \ifvmode\else\enskip\fi
    \ignorespaces
  }%
  {}
\excludecomment{answer}
\includecomment{answer}
           
\begin{document}
\begin{center}
\begin{asy}
size(50mm);
import geometry;

real r = 10; real s = sqrt(r^2/3);
point O = (0, 0); 
point scc = (s, s);
arc qC = arc(circle(O, r), 0, 90);
point C = relpoint(qC, 0); point D = relpoint(qC, 1);
arc sC = arc(circle(scc, s), 135, 315+1e-7); 
point A = relpoint(sC, 0.25); point B = relpoint(sC, 0.75);
point[] ps = intersectionpoints(qC, sC);

draw(O--C);
draw(O--D);
draw(qC);
draw(ps[0]--ps[1]);
draw(sC);
draw(A--B);

perpendicularmark(segment(O,C), segment(O,D), size=2mm, quarter=1);

dot("$O$", O, SW);
dot("$O'$", scc, NE);
dot("$A$", A, W);
dot("$B$", B, S);
//dot("", ps[0]);
//dot("", ps[1]);

\end{asy}
\end{center}

The figure shows a quarter of a circle centred at $O$ and a semicircle centred at $O'$ is inscribed in it. The semicircle touches the quarter of the circle at $A$ and $B$. What is the ratio of the area of $\bigtriangleup OAB$ to the quarter of the circle?

\begin{choices}
\choice $\dfrac{2}{3\pi}$%
\choice $\dfrac{1}{2\pi}$
\choice $\dfrac{3}{5\pi}$
\choice $\dfrac{5}{7\pi}$
\end{choices}

\begin{answer}
\hrulefill\par
\textbf{Ans: A}
\begin{center}
\begin{asy}
size(50mm);
import geometry;

real r = 10; real s = sqrt(r^2/3);
point O = (0, 0); 
point scc = (s, s);
arc qC = arc(circle(O, r), 0, 90);
point C = relpoint(qC, 0); point D = relpoint(qC, 1);
arc sC = arc(circle(scc, s), 135, 315+1e-7); 
point A = relpoint(sC, 0.25); point B = relpoint(sC, 0.75);
point[] ps = intersectionpoints(qC, sC);

draw(O--C);
draw(O--D);
draw(qC);
draw(ps[0]--ps[1]);
draw(sC);
draw(A--B);
draw(O--ps[1], dashed);
draw(O--scc, dashed);
draw(A--scc, dashed);
draw(B--scc, dashed);

perpendicularmark(segment(O,C), segment(O,D), size=2mm, quarter=1);
perpendicularmark(segment(O,B), segment(scc,B), size=2mm, quarter=1);
perpendicularmark(segment(O,A), segment(scc,A), size=2mm, quarter=4);
perpendicularmark(segment(O,scc), segment(scc,D), size=2mm, quarter=3);

dot("$O$", O, SW);
dot("$O'$", scc, NE);
dot("$A$", A, W);
dot("$B$", B, S);
//dot("$E$", ps[0]);
dot("$D$", ps[1], E);

\end{asy}
\end{center}

Let the radius of the quarter of the circle be $r$ and the radius of the semicircle be $s$. Since $OA$ and $OB$ are the tangents to the semicircle, $OAO'B$ is a square.

Consider $\bigtriangleup OBO'$, by Pythagorean theorem,
\begin{equation*}
\begin{aligned}
OB^2 + O'B^2 &= OO'^2 \\
O'A^2 + O'B^2 &= OO'^2 \\
s^2 + s^2 &= OO'^2 \\
OO' &= \sqrt{2}s
\end{aligned}
\end{equation*}
Consider $\bigtriangleup ODO'$, by Pythagorean theorem,
\begin{equation*}
\begin{aligned}
OO'^2 + O'D^2 &= OD^2 \\
2s^2 + s^2 &= r^2 \\
\frac{s^2}{r^2} &= \frac{1}{3}
\end{aligned}
\end{equation*}
\begin{equation*}
\begin{aligned}
\therefore &\quad \frac{\text{area of}\bigtriangleup OAB}{\text{area of the quarter of the circle}} \\
&= \frac{\frac{1}{2}OA\times OB}{\frac{1}{4}\pi r^2} \\
&= \frac{\frac{1}{2}O'B\times O'A}{\frac{1}{4}\pi r^2} \\
&= \frac{\frac{1}{2}s^2}{\frac{1}{4}\pi r^2} \\
&= \frac{2s^2}{\pi r^2} \\
&= \frac{2}{3\pi} \\
\end{aligned}
\end{equation*}
\end{answer}
\end{document}