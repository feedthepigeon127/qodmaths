\documentclass[border=3pt,varwidth=70mm]{standalone}
%standard width: 70 mm
\usepackage[UKenglish]{babel}

\usepackage[utf8]{inputenc}
\usepackage[T1]{fontenc}
\usepackage{libertine}
\usepackage[libertine]{newtxmath}

\usepackage[autostyle]{csquotes}
\usepackage[free-standing-units, binary-units]{siunitx}
\sisetup{%
locale=UK,
range-phrase=--,
table-figures-decimal=2,
table-auto-round=true,
}
\DeclareSIUnit\Molar{\textsc{m}}

\usepackage{chemmacros}
\usepackage{amsmath, amsfonts, amssymb}
\usepackage{textcomp}
\usepackage{bm}
\usepackage{xfrac}

\usepackage{varwidth}
\usepackage{comment}
\makeatletter
\g@addto@macro{\@parboxrestore}{\parskip=\baselineskip}
\makeatother

\usepackage[inline]{asymptote}

\newcounter{choice}
\renewcommand\thechoice{\Alph{choice}}
\newcommand\choicelabel{\thechoice.}

\newenvironment{choices}%
  {\list{\choicelabel}%
     {\usecounter{choice}\def\makelabel##1{\hss\llap{##1}}%
       \settowidth{\leftmargin}{W.\hskip\labelsep\hskip 2.5em}%
       \def\choice{%
         \item
       } % choice
       \labelwidth\leftmargin\advance\labelwidth-\labelsep
       \topsep=0pt
       \partopsep=0pt
     }%
  }%
  {\endlist}

\newenvironment{oneparchoices}%
  {%
    \setcounter{choice}{0}%
    \def\choice{%
      \refstepcounter{choice}%
      \ifnum\value{choice}>1\relax
        \penalty -50\hskip 1em plus 1em\relax
      \fi
      \choicelabel
      \nobreak\enskip
    }% choice
    % If we're continuing the paragraph containing the question,
    % then leave a bit of space before the first choice:
    \ifvmode\else\enskip\fi
    \ignorespaces
  }%
  {}
\excludecomment{answer}
%\includecomment{answer}
           
\begin{document}
\begin{center}
\begin{asy}
size(65mm);
import geometry;
import graph;
import myasy;

real x1=2, x2=4;
point O = (0,0);
point A = (x1, 0);
point B = (x2, 0);

real f(real x) {return -(x-x1)*(x-x2);}
pair F(real x) {return (x, f(x));}
path g = graph(F, 1, 5, operator ..);

point C = ((x1+x2)/2, ((x1+x2)^2-4*x1*x2)/4);
real r = sqrt((x1 - C.x)^2 + C.y^2);
circle cle = circle(C, r);

line[] ts = tangents(cle, O);
point tp = intersectionpoints(cle, ts[1])[0];


draw(Label("$P: y = -x^2 + 6x - 8$", BeginPoint, RightSide), g);
xlimits(-0.5, 5);
ylimits(-3, 3);
xaxis(Label("$x$", EndPoint, NE), YZero(extend=false), Arrow);
yaxis(Label("$y$", EndPoint, NW), XZero(extend=false), Arrow);

xtickmark(x1, size=0.1);
xtickmark(x2, size=0.1);

draw(cle);
draw(O--tp);

label("$O$", O, SW);
label("$A$", A, 2S);
label("$B$", B, 2S);
label("$C$", C, N);
label("$T$", tp, NW);


write(distance2d(O,tp));
write(distance2d(C,tp));
//write(sharpdegrees(line(C,E), line(E,D)));

\end{asy}
\end{center}

In the figure, a parabola and a circle both cut the $x$-axis at point $A$ and $B$. The circle is centred at point $C$, which is the maxima of the parabola. $OT$ is a tangent to the circle from the origin. What is the value of the length of $OT$?  

\begin{choices}
\choice $2$
\choice $2\sqrt{2}$%
\choice $3$
\choice $2\sqrt{3}$
\end{choices}


\begin{answer}
\hrulefill\par
\textbf{Ans: B}

The equation of the parabola is 
\begin{equation*}
\begin{aligned}
-x^2 + 6x - 8 &= 0 \\
-(x-2)(x-4) &= 0 \\
x = 2 &\text{ or } x = 4 \\
\end{aligned}
\end{equation*}
$A=(2,0)$ and $B=(4,0)$. Completing the square, $$-(x-3)^2 + 1$$

$C$ is $(3,1)$. The radius of the circle $r$ is $$r = \sqrt{(3-2)^2+(1-0)^2}=\sqrt{2}$$

\begin{equation*}
\begin{aligned}
\therefore OT^2 + TC^2 &= OC^2 \\
OT^2 &= OC^2 - TC^2 \\
	 &= (3^2+1^2) - 2 \\
\therefore OT &= \sqrt{8} \\
			  &= 2\sqrt{2} \\
\end{aligned}
\end{equation*}

\end{answer}
\end{document}