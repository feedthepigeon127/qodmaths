\documentclass[varwidth=70mm]{standalone}
%standard width: 70 mm
\usepackage[UKenglish]{babel}

\usepackage[utf8]{inputenc}
\usepackage[T1]{fontenc}
\usepackage{libertine}
\usepackage[libertine]{newtxmath}

\usepackage[autostyle]{csquotes}
\usepackage[free-standing-units, binary-units]{siunitx}
\sisetup{%
locale=UK,
range-phrase=--,
table-figures-decimal=2,
table-auto-round=true,
}
\DeclareSIUnit\Molar{\textsc{m}}

\usepackage{chemmacros}
\usepackage{amsmath, amsfonts, amssymb}
\usepackage{textcomp}
\usepackage{bm}
\usepackage{xfrac}

\usepackage{varwidth}
\usepackage{comment}
\makeatletter
\g@addto@macro{\@parboxrestore}{\parskip=\baselineskip}
\makeatother

\usepackage[inline]{asymptote}

\newcounter{choice}
\renewcommand\thechoice{\Alph{choice}}
\newcommand\choicelabel{\thechoice.}

\newenvironment{choices}%
  {\list{\choicelabel}%
     {\usecounter{choice}\def\makelabel##1{\hss\llap{##1}}%
       \settowidth{\leftmargin}{W.\hskip\labelsep\hskip 2.5em}%
       \def\choice{%
         \item
       } % choice
       \labelwidth\leftmargin\advance\labelwidth-\labelsep
       \topsep=0pt
       \partopsep=0pt
     }%
  }%
  {\endlist}

\newenvironment{oneparchoices}%
  {%
    \setcounter{choice}{0}%
    \def\choice{%
      \refstepcounter{choice}%
      \ifnum\value{choice}>1\relax
        \penalty -50\hskip 1em plus 1em\relax
      \fi
      \choicelabel
      \nobreak\enskip
    }% choice
    % If we're continuing the paragraph containing the question,
    % then leave a bit of space before the first choice:
    \ifvmode\else\enskip\fi
    \ignorespaces
  }%
  {}
\excludecomment{answer}
%\includecomment{answer}
           
\begin{document}
\begin{center}
\begin{asy}
size(50mm);
import geometry;

real a = 4; real b = 8; real c = 2; real rdeg = 105;
real r = sqrt((b^2+c^2)*(b^2+(a+c)^2)) / (2*b) ;
point O = (sqrt(r^2 - (a^2/4)), a/2);
point A = rotate(rdeg, O)*(0, 0); 
point B = rotate(rdeg, O)*(0, a); 
point C = rotate(rdeg, O)*(b, a); 
point D = rotate(rdeg, O)*(b, a+c);
circle cle = circle(O, r);
segment AB = segment(A,B);
segment BC = segment(B,C);
segment CD = segment(C,D);

draw(cle);
draw(AB); draw(BC); draw(CD);

dot("$A$", A, SE);
dot("$B$", B, SW);
dot("$C$", C, NE);
dot("$D$", D, NW);
dot("$O$", O);

perpendicularmark(AB, BC, size=2mm, quarter=3);
perpendicularmark(BC, CD, size=2mm, quarter=2);

\end{asy}
\end{center}

In the figure, a circle is centred at $O$ and passes through the point $A$, $B$ and $D$. $AB$ is a chord of the circle. $C$ is a point inside the circle such that $BC\perp AB$ and $CD\perp BC$. If $AB=4$, $BC=8$ and $CD=2$, what is the radius of the circle?

\begin{choices}
\choice $\dfrac{5\sqrt{17}}{4}$%
\choice $\dfrac{5\sqrt{5}}{2}$
\choice $\dfrac{\sqrt{145}}{2}$
\choice $\sqrt{74}$
\end{choices}

\begin{answer}
\hrulefill\par
\textbf{Ans: A}

\begin{center}
\begin{asy}
size(50mm);
import geometry;

real a = 4; real b = 8; real c = 2; real rdeg = 105;
real r = sqrt((b^2+c^2)*(b^2+(a+c)^2)) / (2*b) ;
point O = (sqrt(r^2 - (a^2/4)), a/2);
point A = rotate(rdeg, O)*(0, 0); 
point B = rotate(rdeg, O)*(0, a); 
point C = rotate(rdeg, O)*(b, a); 
point D = rotate(rdeg, O)*(b, a+c);
circle cle = circle(O, r);
segment AB = segment(A,B);
segment BC = segment(B,C);
segment CD = segment(C,D);

point M = (A+B)/2;
point N = intersectionpoint(line(C,D), line(O,M));  
segment OM = segment(O,M);
segment OB = segment(O,B);
segment ON = segment(O,N);
segment CN = segment(C,N);
segment OD = segment(O,D);

draw(cle);
draw(AB); draw(BC); draw(CD);
draw(OM, dashed); draw(OB, dashed);
draw(ON, dashed); draw(OD, dashed); draw(CN, dashed);


dot("$A$", A, SE);
dot("$B$", B, SW);
dot("$C$", C, NE);
dot("$D$", D, NW);
dot("$O$", O);
dot("$M$", M, SE);
dot("$N$", N, NE);

perpendicularmark(AB, BC, size=2mm, quarter=3);
perpendicularmark(BC, CD, size=2mm, quarter=2);
perpendicularmark(OM, AB, size=2mm, quarter=2);
perpendicularmark(ON, CN, size=2mm, quarter=2);

distance("$x$", O, N, 6mm);
distance("$8-x$", O, M, -6mm);

\end{asy}
\end{center}

In the figure, $OM$ is the perpendicular bisector of $AB$ and denote the length of $ON$ by $x$. Let the radius of the circle be $r$.

Consider $\bigtriangleup OBM$, by Pythagorean theorem,
\begin{equation*}
\begin{aligned}
BM^2 + OM^2 &= OB^2 \\
2^2 + (8-x)^2 &= r^2 \\
x^2 - 16x + 68 &= r^2
\end{aligned}
\end{equation*}
Consider $\bigtriangleup OND$, by Pythagorean theorem,
\begin{equation*}
\begin{aligned}
ON^2 + ND^2 &= OD^2 \\
x^2 + (CD+CN)^2 &= r^2 \\
x^2 + (2+2)^2 &= r^2 \\
x &= \sqrt{r^2 - 16}
\end{aligned}
\end{equation*}
$\therefore (r^2 - 16) - 16\sqrt{r^2 - 16} + 68 = r^2$
\begin{equation*}
\begin{aligned}
16\sqrt{r^2 - 16} &= 52 \\
\sqrt{r^2 - 16} &= \frac{13}{4} \\
r^2 - 16 &= \frac{169}{16} \\
r^2 &= \frac{425}{16} \\
	&= \frac{25\cdot 17}{16} \\
\therefore r &= \frac{5\sqrt{17}}{4}
\end{aligned}
\end{equation*}

\end{answer}
\end{document}