\documentclass[border=3pt,varwidth=80mm]{standalone}
%standard width: 70 mm
\usepackage[UKenglish]{babel}

\usepackage[utf8]{inputenc}
\usepackage[T1]{fontenc}
\usepackage{libertine}
\usepackage[libertine]{newtxmath}

\usepackage[autostyle]{csquotes}
\usepackage[free-standing-units, binary-units]{siunitx}
\sisetup{%
locale=UK,
range-phrase=--,
table-figures-decimal=2,
table-auto-round=true,
}
\DeclareSIUnit\Molar{\textsc{m}}

\usepackage{chemmacros}
\usepackage{amsmath, amsfonts, amssymb}
\usepackage{textcomp}
\usepackage{bm}
\usepackage{xfrac}

\usepackage{varwidth}
\usepackage{comment}
\makeatletter
\g@addto@macro{\@parboxrestore}{\parskip=\baselineskip}
\makeatother

\usepackage[inline]{asymptote}

\newcounter{choice}
\renewcommand\thechoice{\Alph{choice}}
\newcommand\choicelabel{\thechoice.}

\newenvironment{choices}%
  {\list{\choicelabel}%
     {\usecounter{choice}\def\makelabel##1{\hss\llap{##1}}%
       \settowidth{\leftmargin}{W.\hskip\labelsep\hskip 2.5em}%
       \def\choice{%
         \item
       } % choice
       \labelwidth\leftmargin\advance\labelwidth-\labelsep
       \topsep=0pt
       \partopsep=0pt
     }%
  }%
  {\endlist}

\newenvironment{oneparchoices}%
  {%
    \setcounter{choice}{0}%
    \def\choice{%
      \refstepcounter{choice}%
      \ifnum\value{choice}>1\relax
        \penalty -50\hskip 1em plus 1em\relax
      \fi
      \choicelabel
      \nobreak\enskip
    }% choice
    % If we're continuing the paragraph containing the question,
    % then leave a bit of space before the first choice:
    \ifvmode\else\enskip\fi
    \ignorespaces
  }%
  {}
\excludecomment{answer}
\includecomment{answer}
           
\begin{document}
\begin{center}
\begin{asy}
size(65mm);
import geometry;

real r = 1;

point O = (0, 0);
point D = (r, 0);
point A = r*dir(-108);
point B = rotate(-108,A)*(A+r*unit(O-A));
point C = rotate(-108,B)*(B+r*unit(A-B));
point E = O+r*unit(O-A);
point K = intersectionpoint(line(B,C), line(O,D));

path cle = scale(r)*unitcircle;

draw(cle);
draw(O--A--B--C--D--cycle);
draw(D--K);
draw(C--K);
draw(O--E);
draw(E--K);

label("$O$", O, W);
label("$A$", A, (unit(A-O)+unit(A-B))/2);
label("$B$", B, (unit(B-A)+unit(B-C))/2);
label("$C$", C, unit(C-D));
label("$D$", D, unit(D-C));
label("$E$", E, (unit(E-O)+unit(E-K))/2);
label("$K$", K, (unit(K-C)+unit(K-E))/2);

markangle("", E,K,O);

//write(degrees(line(E,K),line(O,K)));

\end{asy}
\end{center}

In the figure, $OABCD$ is a regular pentagon. $O$ is the centre and $AE$ is the diameter of the circle $ADE$. $BC$ produced and $OD$ produced meet at the point $K$. Find $\angle EKO$ in nearest degree.

\begin{choices}
\choice \ang{17}
\choice \ang{20}
\choice \ang{22}%
\choice \ang{25}
\end{choices}


\begin{answer}
\hrulefill\par
\textbf{Ans: C}

Denote the radius of the circle by $r$.

Interior angles of regular pentagon are \ang{108}, so $\angle KDC = \angle KCD = \ang{180} - \ang{108} = \ang{72}$, and $\angle CKD = \ang{36}$.

Using Sine law in $\bigtriangleup CDK$,
\begin{equation*}
\begin{aligned}
\frac{KD}{\sin\ang{72}} &= \frac{CD}{\sin\ang{36}} \\
KD &= r\frac{\sin\ang{72}}{\sin\ang{36}} \\
\end{aligned}
\end{equation*}

Consider $\bigtriangleup EOK$, $\angle EOK = \ang{180}-\ang{108} = \ang{72}$. Using Cosine law,
\begin{equation*}
\begin{aligned}
EK^2 &= OE^2 + OK^2 - 2(OE)(OK)\cos\angle EOK \\
&= r^2 + (r+r\frac{\sin\ang{72}}{\sin\ang{36}})^2 - 2r(r+r\frac{\sin\ang{72}}{\sin\ang{36}})\cos\ang{72} \\
EK &= r\sqrt{1 + (1+\frac{\sin\ang{72}}{\sin\ang{36}})^2 - 2(1+\frac{\sin\ang{72}}{\sin\ang{36}})\cos\ang{72}} \\
\end{aligned}
\end{equation*}

Using Sine law,
\begin{equation*}
\begin{aligned}
\frac{OE}{\sin\angle EKO} &= \frac{EK}{\sin\angle EOK} \\
\sin\angle EKO &= \frac{r\sin\ang{72}}{r\sqrt{1 + (1+\frac{\sin\ang{72}}{\sin\ang{36}})^2 - 2(1+\frac{\sin\ang{72}}{\sin\ang{36}})\cos\ang{72}}} \\
\angle EKO &= \ang{22.39} \\
\end{aligned}
\end{equation*}

\end{answer}
\end{document}