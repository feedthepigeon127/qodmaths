\documentclass[border=3pt,varwidth=70mm]{standalone}
%standard width: 70 mm
\usepackage[UKenglish]{babel}

\usepackage[utf8]{inputenc}
\usepackage[T1]{fontenc}
\usepackage{libertine}
\usepackage[libertine]{newtxmath}

\usepackage[autostyle]{csquotes}
\usepackage[free-standing-units, binary-units]{siunitx}
\sisetup{%
locale=UK,
range-phrase=--,
table-figures-decimal=2,
table-auto-round=true,
}
\DeclareSIUnit\Molar{\textsc{m}}

\usepackage{chemmacros}
\usepackage{amsmath, amsfonts, amssymb}
\usepackage{textcomp}
\usepackage{bm}
\usepackage{xfrac}

\usepackage{varwidth}
\usepackage{comment}
\makeatletter
\g@addto@macro{\@parboxrestore}{\parskip=\baselineskip}
\makeatother

\usepackage[inline]{asymptote}

\newcounter{choice}
\renewcommand\thechoice{\Alph{choice}}
\newcommand\choicelabel{\thechoice.}

\newenvironment{choices}%
  {\list{\choicelabel}%
     {\usecounter{choice}\def\makelabel##1{\hss\llap{##1}}%
       \settowidth{\leftmargin}{W.\hskip\labelsep\hskip 2.5em}%
       \def\choice{%
         \item
       } % choice
       \labelwidth\leftmargin\advance\labelwidth-\labelsep
       \topsep=0pt
       \partopsep=0pt
     }%
  }%
  {\endlist}

\newenvironment{oneparchoices}%
  {%
    \setcounter{choice}{0}%
    \def\choice{%
      \refstepcounter{choice}%
      \ifnum\value{choice}>1\relax
        \penalty -50\hskip 1em plus 1em\relax
      \fi
      \choicelabel
      \nobreak\enskip
    }% choice
    % If we're continuing the paragraph containing the question,
    % then leave a bit of space before the first choice:
    \ifvmode\else\enskip\fi
    \ignorespaces
  }%
  {}
\excludecomment{answer}
%\includecomment{answer}
           
\begin{document}
\begin{center}
\begin{asy}
size(65mm);
import geometry;

real p = 1; real q = 4; real r = 2;
real g = 5; real h = 7;

point B = (0, 0);
point C = (p+q+r, 0);
point A = (g, h);
point E = C + (q, 0);
point D = A + (p+q+r, 0);
point F = A + (p,0);
point G = A + (p+q, 0);
point X = intersectionpoint(line(C,D), line(G,E));

draw(A--B--C--D--cycle);
draw(C--E);
draw(F--C);
draw(G--E);

label("$A$", A, (unit(A-B)+unit(A-D))/2);
label("$B$", B, (unit(B-A)+unit(B-E))/2);
label("$C$", C, (unit(C-D)+unit(C-F))/2);
label("$D$", D, (unit(D-C)+unit(D-A))/2);
label("$E$", E, (unit(E-B)+unit(E-G))/2);
label("$F$", F, unit(F-C));
label("$G$", G, unit(G-E));
label("$X$", X, (unit(X-G)+unit(X-C))/2);

\end{asy}
\end{center}

In the figure, $ABCD$ is a parallelogram. $F$ and $G$ are points on $AD$ such that $FCEG$ is a parallelogram. $GE$ and $DC$ intersect at point $X$. If $AF:FG:GD=1:4:2$ and the area of $\bigtriangleup CXE$ is \SI{28}{\centi\meter\squared}, then the area of the trapezium $ABCF$ is 

\begin{choices}
\choice \SI{63}{\centi\meter\squared}.
\choice \SI{77}{\centi\meter\squared}.
\choice \SI{84}{\centi\meter\squared}.%
\choice \SI{88}{\centi\meter\squared}.
\end{choices}


\begin{answer}
\hrulefill\par
\textbf{Ans: C}

$\bigtriangleup CXE\sim \bigtriangleup DXG$, so
\begin{equation*}
\begin{aligned}
\frac{\bigtriangleup DXG}{\bigtriangleup CXE} &= \left(\frac{GD}{EC}\right)^2 \\
											&= \left(\frac{GD}{FG}\right)^2 \\
\bigtriangleup DXG &= \bigtriangleup CXE \left(\frac{GD}{FG}\right)^2 \\
					&= 28\left(\frac{2}{4}\right)^2 \\
					&= \SI{7}{\centi\meter\squared} \\
\end{aligned}
\end{equation*}

$\bigtriangleup DXG\sim \bigtriangleup DCF$, so
\begin{equation*}
\begin{aligned}
\frac{\bigtriangleup DCF}{\bigtriangleup DXG} &= \left(\frac{FD}{GD}\right)^2 \\
											&= \left(\frac{FG+GD}{GD}\right)^2 \\
\bigtriangleup DCF &= \bigtriangleup DXG \left(\frac{FG+GD}{GD}\right)^2 \\
					&= 7\left(\frac{4+2}{2}\right)^2 \\
					&= \SI{63}{\centi\meter\squared} \\
\end{aligned}
\end{equation*}

$\bigtriangleup DCF$ and parallelogram $ABCD$ share the same height, so
\begin{equation*}
\begin{aligned}
\frac{\diamond ABCD}{\bigtriangleup DCF} &= \frac{AD}{\frac{1}{2}FD} \\
											&= \frac{2AD}{FD} \\
\diamond ABCD &= \bigtriangleup DCF\frac{2AD}{FD} \\
			  &= 63\frac{2\times 7}{6} \\
			  &= \SI{147}{\centi\meter\squared} \\
\end{aligned}
\end{equation*}

\begin{equation*}
\begin{aligned}
\text{Area of trapezium } ABCF &= \diamond ABCD - \bigtriangleup DCF \\
						&= 147 - 63 \\
						&= \SI{84}{\centi\meter\squared} \\  
\end{aligned}
\end{equation*}

\end{answer}
\end{document}