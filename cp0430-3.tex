\documentclass[border=3pt,varwidth=70mm]{standalone}
%standard width: 70 mm
\usepackage[UKenglish]{babel}

\usepackage[utf8]{inputenc}
\usepackage[T1]{fontenc}
\usepackage{libertine}
\usepackage[libertine]{newtxmath}

\usepackage[autostyle]{csquotes}
\usepackage[free-standing-units, binary-units]{siunitx}
\sisetup{%
locale=UK,
range-phrase=--,
table-figures-decimal=2,
table-auto-round=true,
}
\DeclareSIUnit\Molar{\textsc{m}}

\usepackage{chemmacros}
\usepackage{amsmath, amsfonts, amssymb}
\usepackage{textcomp}
\usepackage{bm}
\usepackage{xfrac}

\usepackage{varwidth}
\usepackage{comment}
\makeatletter
\g@addto@macro{\@parboxrestore}{\parskip=\baselineskip}
\makeatother

\usepackage[inline]{asymptote}

\newcounter{choice}
\renewcommand\thechoice{\Alph{choice}}
\newcommand\choicelabel{\thechoice.}

\newenvironment{choices}%
  {\list{\choicelabel}%
     {\usecounter{choice}\def\makelabel##1{\hss\llap{##1}}%
       \settowidth{\leftmargin}{W.\hskip\labelsep\hskip 2.5em}%
       \def\choice{%
         \item
       } % choice
       \labelwidth\leftmargin\advance\labelwidth-\labelsep
       \topsep=0pt
       \partopsep=0pt
     }%
  }%
  {\endlist}

\newenvironment{oneparchoices}%
  {%
    \setcounter{choice}{0}%
    \def\choice{%
      \refstepcounter{choice}%
      \ifnum\value{choice}>1\relax
        \penalty -50\hskip 1em plus 1em\relax
      \fi
      \choicelabel
      \nobreak\enskip
    }% choice
    % If we're continuing the paragraph containing the question,
    % then leave a bit of space before the first choice:
    \ifvmode\else\enskip\fi
    \ignorespaces
  }%
  {}
\excludecomment{answer}
%\includecomment{answer}
           
\begin{document}
\begin{center}
\begin{asy}
size(65mm);
import geometry;
import myasy;

real r = 1; real alpha = 10; 
real l1 = 0.5;

point O = (0,0);

circle cle = circle((point)(0,0), r);
point[] top = intersectionpoints(cle, line((point)(0,l1),(point)(2,l1)));
point A = top[0];
point B = top[1];
point D = intersectionpoints(cle, line((point)(A.x,10), A))[0];
point C = intersectionpoints(cle, line((point)(B.x,10), B))[0];

point E = intersectionpoint(line(D,B), line(A, A+dir(alpha)));
point F = intersectionpoints(cle, line(E, A))[1];

draw(cle);
draw(A--E);
draw(A--D);
draw(D--E);
draw(A--C);
draw(O--F);
draw(C--E);

label("$O$", O, labelat(O, new pair[] {A,E}));
label("$A$", A, labelat(A, new pair[] {O,D,F}));
//label("$B$", B);
label("$D$", C, labelat(C, new pair[] {O,E}));
label("$B$", D, labelat(D, new pair[] {A,O}));
label("$C$", E, labelat(E, new pair[] {A,O,C}));
label("$E$", F, labelat(F, new pair[] {O}));

write(distance2d(A,E));
write(distance2d(C,F));
write(distance2d(F,E));
write(sharpdegrees(line(F,O), line(B,O)));
write(sharpdegrees(line(A,O), line(D,A)));
write(sharpdegrees(line(F,C), line(E,A)));

\end{asy}
\end{center}

In the figure, a circle is centred at point $O$ and point $A$, $B$, $D$ and $E$ are lying on the circle. Point $C$ is outside the circle such that $AEC$ and $BOC$ are straight lines. $\bigtriangleup ABO$ is an equilateral triangle. $AD$ is the diameter of the circle. If $\angle COE = \ang{20}$ and $AD=\SI{6}{\centi\meter}$, then the area of $\bigtriangleup ACD$ correct to the nearest \SI{0.1}{\centi\meter\squared} is 

\begin{choices}
\choice \SI{8.9}{\centi\meter\squared}
\choice \SI{14.6}{\centi\meter\squared}%
\choice \SI{15.6}{\centi\meter\squared}
\choice \SI{20.4}{\centi\meter\squared}
\end{choices}


\begin{answer}
\hrulefill\par
\textbf{Ans: B}
\begin{center}
\begin{asy}
size(65mm);
import geometry;
import myasy;

real r = 1; real alpha = 10; 
real l1 = 0.5;

point O = (0,0);

circle cle = circle((point)(0,0), r);
point[] top = intersectionpoints(cle, line((point)(0,l1),(point)(2,l1)));
point A = top[0];
point B = top[1];
point D = intersectionpoints(cle, line((point)(A.x,10), A))[0];
point C = intersectionpoints(cle, line((point)(B.x,10), B))[0];

point E = intersectionpoint(line(D,B), line(A, A+dir(alpha)));
point F = intersectionpoints(cle, line(E, A))[1];

draw(cle);
draw(A--E);
draw(A--D);
draw(D--E);
draw(A--C);
draw(O--F);
draw(C--E);

draw(B--A, dashed);
draw(C--F, dashed);

label("$O$", O, labelat(O, new pair[] {A,E}));
label("$A$", A, labelat(A, new pair[] {O,D,F}));
label("$F$", B, S);
label("$D$", C, labelat(C, new pair[] {O,E}));
label("$B$", D, labelat(D, new pair[] {A,O}));
label("$C$", E, labelat(E, new pair[] {A,O,C}));
label("$E$", F, labelat(F, new pair[] {O}));

markrightangle(C, F, A);
markrightangle(D, A, B);

\end{asy}
\end{center}

$\angle EAF = \ang{10}$ (angle subtended at the circumference is half the angle subtended at the centre for the same arc)

Consider $\bigtriangleup ABC$,
$$\angle ACB = \ang{180}-\ang{10}-\ang{90}-\ang{60}=\ang{20}$$

Since $\angle ACB = \angle COE = \ang{20}$, $EC=OE=\SI{3}{\centi\meter}$.

$\angle EAD = \ang{10} + (\ang{90}-\ang{60}) = \ang{40}$, consider $\bigtriangleup AED$, $$AE = 6\cos\ang{40}$$ and $$ED = 6\sin\ang{40}$$

\begin{equation*}
\begin{aligned}
\therefore\bigtriangleup ACD &= \frac{1}{2}\times AC\times ED \\
						   &= \frac{1}{2}\left(6\cos\ang{40}+3\right)\left(6\sin\ang{40}\right) \\
						   &= \SI{14.648}{\centi\meter\squared}
\end{aligned}
\end{equation*}
\end{answer}
\end{document}