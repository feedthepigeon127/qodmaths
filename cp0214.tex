\documentclass[varwidth=70mm]{standalone}
%standard width: 70 mm
\usepackage[UKenglish]{babel}

\usepackage[utf8]{inputenc}
\usepackage[T1]{fontenc}
\usepackage{libertine}
\usepackage[libertine]{newtxmath}

\usepackage[autostyle]{csquotes}
\usepackage[free-standing-units, binary-units]{siunitx}
\sisetup{%
locale=UK,
range-phrase=--,
table-figures-decimal=2,
table-auto-round=true,
}
\DeclareSIUnit\Molar{\textsc{m}}

\usepackage{chemmacros}
\usepackage{amsmath, amsfonts, amssymb}
\usepackage{textcomp}
\usepackage{bm}
\usepackage{xfrac}

\usepackage{varwidth}
\usepackage{comment}
\makeatletter
\g@addto@macro{\@parboxrestore}{\parskip=\baselineskip}
\makeatother

\usepackage[inline]{asymptote}

\newcounter{choice}
\renewcommand\thechoice{\Alph{choice}}
\newcommand\choicelabel{\thechoice.}

\newenvironment{choices}%
  {\list{\choicelabel}%
     {\usecounter{choice}\def\makelabel##1{\hss\llap{##1}}%
       \settowidth{\leftmargin}{W.\hskip\labelsep\hskip 2.5em}%
       \def\choice{%
         \item
       } % choice
       \labelwidth\leftmargin\advance\labelwidth-\labelsep
       \topsep=0pt
       \partopsep=0pt
     }%
  }%
  {\endlist}

\newenvironment{oneparchoices}%
  {%
    \setcounter{choice}{0}%
    \def\choice{%
      \refstepcounter{choice}%
      \ifnum\value{choice}>1\relax
        \penalty -50\hskip 1em plus 1em\relax
      \fi
      \choicelabel
      \nobreak\enskip
    }% choice
    % If we're continuing the paragraph containing the question,
    % then leave a bit of space before the first choice:
    \ifvmode\else\enskip\fi
    \ignorespaces
  }%
  {}
\excludecomment{answer}
\includecomment{answer}
           
\begin{document}
\begin{center}
\begin{asy}
size(50mm);
import graph;

real f(real x) {return (2-sin(x))^2 + 1;}
pair F(real x) {return (f(x), x);}

draw(graph(F, -2*pi, 2*pi, operator ..), black);
draw(Label("$x$", EndPoint), (-1,0)--(12,0), Arrow(1mm));
draw(Label("$y$", EndPoint), (0,-2*pi)--(0,2*pi), Arrow(1mm));

\end{asy}
\end{center}

The figure shows the graph of $\sin y = 2-\sqrt{x-1}$. Which of the following must be true?
\begin{itemize}
	\item[I.] the graph is bound by the lines $x=2$ and $x=10$
	\item[II.] the graph intersects the $x$-axis at $(5, 0)$
	\item[III.] $y=1$ is an axis of symmetry
\end{itemize} 

\begin{choices}
\choice I and II only%
\choice I and III only
\choice II and III only
\choice I, II and III
\end{choices}

\begin{answer}
\hrulefill\par
\textbf{Ans: A}

$\because -1 \le \sin y \le 1$
\begin{equation*}
\therefore\begin{cases}
2-\sqrt{x-1} = 1 \\
2-\sqrt{x-1} = -1 
\end{cases}
\end{equation*}
\begin{equation*}
\begin{cases}
\sqrt{x-1} = 1 \\
\sqrt{x-1} = 3
\end{cases}
\end{equation*}
\begin{equation*}
\begin{cases}
x = 2 \\
x = 10
\end{cases}
\end{equation*}
The minimum and maximum of the graph is $x=2$ and $x=10$ respectively, so it is bound by $x=2$ and $x=10$.

$\therefore$ I is true

If it intersects the $x$-axis, $y=0$
\begin{equation*}
\begin{aligned}
\therefore \sin(0) &= 2-\sqrt{x-1} \\
0 &= 2-\sqrt{x-1} \\
x &= 5 
\end{aligned}
\end{equation*}
$\therefore$ II is true

Only horizontal lines cutting through the extrema of the graph will be axes of symmetry.
\begin{equation*}
\therefore\begin{cases}
\sin y = 2-\sqrt{2-1} \\
\sin y = 2-\sqrt{10-1} 
\end{cases}
\end{equation*}
\begin{equation*}
\begin{cases}
\sin y = 1 \\
\sin y = -1 
\end{cases}
\end{equation*}
As $y=1$ does not satisfies the above equations, it is not an axis of symmetry.

$\therefore$ III is false
\end{answer}
\end{document}