\documentclass[varwidth=70mm]{standalone}
%standard width: 70 mm
\usepackage[UKenglish]{babel}

\usepackage[utf8]{inputenc}
\usepackage[T1]{fontenc}
\usepackage{libertine}
\usepackage[libertine]{newtxmath}

\usepackage[autostyle]{csquotes}
\usepackage[free-standing-units, binary-units]{siunitx}
\sisetup{%
locale=UK,
range-phrase=--,
table-figures-decimal=2,
table-auto-round=true,
}
\DeclareSIUnit\Molar{\textsc{m}}

\usepackage{chemmacros}
\usepackage{amsmath, amsfonts, amssymb}
\usepackage{textcomp}
\usepackage{bm}
\usepackage{xfrac}

\usepackage{varwidth}
\usepackage{comment}
\makeatletter
\g@addto@macro{\@parboxrestore}{\parskip=\baselineskip}
\makeatother

\usepackage[inline]{asymptote}

\newcounter{choice}
\renewcommand\thechoice{\Alph{choice}}
\newcommand\choicelabel{\thechoice.}

\newenvironment{choices}%
  {\list{\choicelabel}%
     {\usecounter{choice}\def\makelabel##1{\hss\llap{##1}}%
       \settowidth{\leftmargin}{W.\hskip\labelsep\hskip 2.5em}%
       \def\choice{%
         \item
       } % choice
       \labelwidth\leftmargin\advance\labelwidth-\labelsep
       \topsep=0pt
       \partopsep=0pt
     }%
  }%
  {\endlist}

\newenvironment{oneparchoices}%
  {%
    \setcounter{choice}{0}%
    \def\choice{%
      \refstepcounter{choice}%
      \ifnum\value{choice}>1\relax
        \penalty -50\hskip 1em plus 1em\relax
      \fi
      \choicelabel
      \nobreak\enskip
    }% choice
    % If we're continuing the paragraph containing the question,
    % then leave a bit of space before the first choice:
    \ifvmode\else\enskip\fi
    \ignorespaces
  }%
  {}
\excludecomment{answer}
%\includecomment{answer}
           
\begin{document}
\begin{center}
\begin{asy}
size(50mm);
import geometry;

point A = (2, 1); point B = (5, 4); 
point C = (0, 2*sqrt(5)-1); point P = (7-2*sqrt(5), 2*sqrt(5)-1);
circle cle = circle(A, B, C);

draw(Label("$x$", EndPoint), (-1,0)--(7,0), Arrow(1mm));
draw(Label("$y$", EndPoint), (0,-1)--(0,7), Arrow(1mm));
draw(cle);

dot("$A$", A, SW);
dot("$B$", B, NE);
dot("$O$", P, E);
\end{asy}
\end{center}

The figure shows a circle centred at $O$ that passes through the point $A$ and $B$ and tangent to the $y$-axis. If $A=(2, 1)$ and $B=(5, 4)$, what is the radius of the circle? 
\begin{choices}
\choice $2\sqrt{5}-2$
\choice $7-2\sqrt{5}$%
\choice $2\sqrt{5}+2$
\choice $7+2\sqrt{5}$
\end{choices}

\begin{answer}
\hrulefill\par
\textbf{Ans: B}

\begin{center}
\begin{asy}
size(50mm);
import geometry;

point A = (2, 1); point B = (5, 4); 
point C = (0, 2*sqrt(5)-1); point P = (7-2*sqrt(5), 2*sqrt(5)-1);
circle cle = circle(A, B, C);

draw(Label("$x$", EndPoint), (-1,0)--(7,0), Arrow(1mm));
draw(Label("$y$", EndPoint), (0,-1)--(0,7), Arrow(1mm));
draw(cle);

dot("", A, SW);
dot("", B, NE);
dot("$(p, q)$", P, SW);

segment AB = segment(A, B);
line perpbis = perpendicular(P, AB);

draw(AB, StickIntervalMarker(2, 1, black)); 
draw(Label("$y=-x+6$", EndPoint, SW), perpbis);
perpendicularmark(AB, perpbis, quarter=1, size=5);

\end{asy}
\end{center}
Let the centre of the circle be $(p, q)$ and its radius $r$. As the circle is tangent to the $y$-axis, $p=r$, so the centre of the circle can be denoted by $(r,q)$. 

A perpendicular bisector of a chord always passes through the centre of the circle. The perpendicular bisector of the chord $AB$ is $y=-x+6$, so we have
\begin{equation*}
\begin{aligned}
q &= -p + 6 \\
q &= -r + 6
\end{aligned}
\end{equation*}

The equation of the circle that passes through $(2,1)$ with a centre $(r, q)$ is
\begin{equation*}
\begin{aligned}
(2-r)^2 + (1-q)^2 &= r^2 \\
(2-r)^2 + \left[1-(-r+6)\right]^2 &= r^2 \\
(2-r)^2 + (r-5)^2 &= r^2 \\
4-4r+r^2+r^2-10r+25 &= r^2 \\
r^2-14r+29 &= 0 \\
r = 7\pm 2\sqrt{5}
\end{aligned}
\end{equation*}
Since $7+2\sqrt{5}>5$, $r=7-2\sqrt{5}$
\end{answer}
\end{document}


