\documentclass[border=3pt,varwidth=70mm]{standalone}
%standard width: 70 mm
\usepackage[UKenglish]{babel}

\usepackage[utf8]{inputenc}
\usepackage[T1]{fontenc}
\usepackage{libertine}
\usepackage[libertine]{newtxmath}

\usepackage[autostyle]{csquotes}
\usepackage[free-standing-units, binary-units]{siunitx}
\sisetup{%
locale=UK,
range-phrase=--,
table-figures-decimal=2,
table-auto-round=true,
}
\DeclareSIUnit\Molar{\textsc{m}}

\usepackage{chemmacros}
\usepackage{amsmath, amsfonts, amssymb}
\usepackage{textcomp}
\usepackage{bm}
\usepackage{xfrac}

\usepackage{varwidth}
\usepackage{comment}
\makeatletter
\g@addto@macro{\@parboxrestore}{\parskip=\baselineskip}
\makeatother

\usepackage[inline]{asymptote}

\newcounter{choice}
\renewcommand\thechoice{\Alph{choice}}
\newcommand\choicelabel{\thechoice.}

\newenvironment{choices}%
  {\list{\choicelabel}%
     {\usecounter{choice}\def\makelabel##1{\hss\llap{##1}}%
       \settowidth{\leftmargin}{W.\hskip\labelsep\hskip 2.5em}%
       \def\choice{%
         \item
       } % choice
       \labelwidth\leftmargin\advance\labelwidth-\labelsep
       \topsep=0pt
       \partopsep=0pt
     }%
  }%
  {\endlist}

\newenvironment{oneparchoices}%
  {%
    \setcounter{choice}{0}%
    \def\choice{%
      \refstepcounter{choice}%
      \ifnum\value{choice}>1\relax
        \penalty -50\hskip 1em plus 1em\relax
      \fi
      \choicelabel
      \nobreak\enskip
    }% choice
    % If we're continuing the paragraph containing the question,
    % then leave a bit of space before the first choice:
    \ifvmode\else\enskip\fi
    \ignorespaces
  }%
  {}
\excludecomment{answer}
%\includecomment{answer}
           
\begin{document}
\begin{center}
\begin{asy}
size(65mm);
import geometry;
import myasy;

real l = 10;

point A = (0,0);
point B = A + (l,0);
point C = A + rotate(60,A)*(l,0);
point D = A + (l/2,0);
point E = D + (0,l/2);
point F = A + (0,l/2);
point X = intersectionpoint(line(A,C),line(E,F));

draw(A--B--C--cycle);
draw(A--D--E--F--cycle);
draw(C--E);

label("$A$", A, labelat(A, new pair[] {B,C,F}));
label("$B$", B, labelat(B, new pair[] {A,C}));
label("$C$", C, labelat(C, new pair[] {A,C}));
label("$D$", D, labelat(D, new pair[] {E}));
label("$E$", E, labelat(E, new pair[] {F,D}));
label("$F$", F, labelat(F, new pair[] {A,E}));
label("$X$", X, labelat(X, new pair[] {A,E}));

write(distance2d(F,X));
//write(sharpdegrees(line(C,E), line(E,D)));

\end{asy}
\end{center}

In the figure, $\bigtriangleup ABC$ is an equilateral triangle and $ADEF$ is a square. $EF$ and $AC$ intersect at point $X$. $CED$ is a straight line. If $AB=\SI{10}{\centi\meter}$, then the area of $\bigtriangleup AFX$ correct to the nearest \SI{0.1}{\centi\meter\squared} is 

\begin{choices}
\choice \SI{4.8}{\centi\meter\squared}.
\choice \SI{7.2}{\centi\meter\squared}.
\choice \SI{8.8}{\centi\meter\squared}.
\choice \SI{10.8}{\centi\meter\squared}.
\end{choices}


\begin{answer}
\hrulefill\par
\textbf{Ans: B}

$CD$ is the altitude of the equilateral triangle, so $AD = \SI{5}{\centi\meter}$.

$CE=CD-ED$ and $CD=AC\sin\ang{60}=5\sqrt{3}$,
\begin{equation*}
\begin{aligned}
\therefore CE &= 5\sqrt{3} - 5 \\
			  &= 5(\sqrt{3}-1) \\
\end{aligned}
\end{equation*}
\begin{equation*}
\begin{aligned}
\because \bigtriangleup CXE &\sim \bigtriangleup CAD \\
\therefore \frac{XE}{AD} &= \frac{CE}{CD} \\
XE &= 5\times\frac{5(\sqrt{3}-1)}{5\sqrt{3}} \\
	&= \frac{5(\sqrt{3}-1)}{\sqrt{3}} \\
\end{aligned}
\end{equation*}
\begin{equation*}
\begin{aligned}
\therefore \bigtriangleup AFX &= \frac{AF\times FX}{2} \\
							  &= \frac{5\times(5-\frac{5(\sqrt{3}-1)}{\sqrt{3}})}{2} \\
							  &= \SI{7.217}{\centi\meter\squared}
\end{aligned}
\end{equation*}

\end{answer}
\end{document}