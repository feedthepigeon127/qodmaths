\documentclass[varwidth=70mm]{standalone}
%standard width: 70 mm
\usepackage[UKenglish]{babel}

\usepackage[utf8]{inputenc}
\usepackage[T1]{fontenc}
\usepackage{libertine}
\usepackage[libertine]{newtxmath}

\usepackage[autostyle]{csquotes}
\usepackage[free-standing-units, binary-units]{siunitx}
\sisetup{%
locale=UK,
range-phrase=--,
table-figures-decimal=2,
table-auto-round=true,
}
\DeclareSIUnit\Molar{\textsc{m}}

\usepackage{chemmacros}
\usepackage{amsmath, amsfonts, amssymb}
\usepackage{textcomp}
\usepackage{bm}
\usepackage{xfrac}

\usepackage{varwidth}
\usepackage{comment}
\makeatletter
\g@addto@macro{\@parboxrestore}{\parskip=\baselineskip}
\makeatother

\usepackage[inline]{asymptote}

\newcounter{choice}
\renewcommand\thechoice{\Alph{choice}}
\newcommand\choicelabel{\thechoice.}

\newenvironment{choices}%
  {\list{\choicelabel}%
     {\usecounter{choice}\def\makelabel##1{\hss\llap{##1}}%
       \settowidth{\leftmargin}{W.\hskip\labelsep\hskip 2.5em}%
       \def\choice{%
         \item
       } % choice
       \labelwidth\leftmargin\advance\labelwidth-\labelsep
       \topsep=0pt
       \partopsep=0pt
     }%
  }%
  {\endlist}

\newenvironment{oneparchoices}%
  {%
    \setcounter{choice}{0}%
    \def\choice{%
      \refstepcounter{choice}%
      \ifnum\value{choice}>1\relax
        \penalty -50\hskip 1em plus 1em\relax
      \fi
      \choicelabel
      \nobreak\enskip
    }% choice
    % If we're continuing the paragraph containing the question,
    % then leave a bit of space before the first choice:
    \ifvmode\else\enskip\fi
    \ignorespaces
  }%
  {}
\excludecomment{answer}
%\includecomment{answer}
           
\begin{document}
\begin{center}
\begin{asy}
size(50mm);
import geometry;

//show(currentcoordsys);
real a = 4; 
path sqr = scale(a)*unitsquare;
point A = relpoint(sqr, 0.75);
point B = relpoint(sqr, 0);
point C = relpoint(sqr, 0.25);
point D = relpoint(sqr, 0.5);
point scc = (a/2, a);
circle cle = circle(scc, a/2);
arc sc = arc(cle, 180, 360);

line[] tC = tangents(cle, C);
point E_ = intersectionpoint(tC[0], segment(A,B));
segment dC = segment(C, E_);

line[] tB = tangents(cle, B);
point F = intersectionpoint(tB[0], segment(C,D));
segment dB = segment(B, F);

point P = intersectionpoint(dC, dB);

draw(sqr);
draw(sc);
draw(dC);
draw(dB);

label("$A$", A, NW);
label("$B$", B, SW);
label("$C$", C, SE);
label("$D$", D, NE);
label("$E$", E_, W);
label("$F$", F, E);
label("$P$", P, S);
//dot("", P);

\end{asy}
\end{center}

In the figure, a semicircle with diameter $AD$ is inscribed in the square $ABCD$. Two lines, $CE$ and $BF$, are tangents to the semicircle, and they intersect at $P$. If $BC=4$, what is the area of $\bigtriangleup BPC$? 

\begin{choices}
\choice $2\sqrt{2}$
\choice $3$%
\choice $\sqrt{10}$
\choice $2\sqrt{3}$
\end{choices}

\begin{answer}
\hrulefill\par
\textbf{Ans: B}

\begin{center}
\begin{asy}
size(50mm);
import geometry;

//show(currentcoordsys);
real a = 4; 
path sqr = scale(a)*unitsquare;
point A = relpoint(sqr, 0.75);
point B = relpoint(sqr, 0);
point C = relpoint(sqr, 0.25);
point D = relpoint(sqr, 0.5);
point scc = (a/2, a);
circle cle = circle(scc, a/2);
arc sc = arc(cle, 180, 360);

line[] tC = tangents(cle, C);
point E_ = intersectionpoint(tC[0], segment(A,B));
segment dC = segment(C, E_);
point X = tC[0].B ;

line[] tB = tangents(cle, B);
point F = intersectionpoint(tB[0], segment(C,D));
segment dB = segment(B, F);

point P = intersectionpoint(dC, dB);

point M = (a/2, 0);

draw(sqr);
draw(sc);
draw(dC);
draw(dB);
draw(P--M, dashed);

label("$A$", A, NW);
label("$B$", B, SW);
label("$C$", C, SE);
label("$D$", D, NE);
label("$E$", E_, W);
label("$F$", F, E);
label("$P$", P, S);
dot("$X$", X, NE);
dot("$M$", M, S);

perpendicularmark(segment(B,C), segment(A,B), size=2mm, quarter=1);
perpendicularmark(segment(P,M), segment(B,C), size=2mm, quarter=2);

\end{asy}
\end{center}

$X$ denotes the point when $CE$ touches the semicircle. As the lengths of tangent to a circle are the same when they are drawn from the same point, $CX=CD=4$ and $EX=EA$. The length of $EA$ is denoted by $x$, then by Pythagorean theorem,
\begin{equation*}
\begin{aligned}
BC^2 + BE^2 &= CE^2 \\
4^2 + (4-x)^2 &= (4+x)^2 \\
16 + 16 - 8x + x^2 &= 16 + 8x + x^2 \\
x &= 1
\end{aligned}
\end{equation*}
Consider $\bigtriangleup CEB$,
\begin{equation*}
\begin{aligned}
\tan\angle BCE &= \frac{EB}{BC} \\
			   &= \frac{4-1}{4} \\
			   &= \frac{3}{4} \\
\end{aligned}
\end{equation*}
Consider $\bigtriangleup CPM$,
\begin{equation*}
\begin{aligned}
\tan\angle MCP &= \frac{PM}{MC} \\
\tan\angle BCE &= \frac{PM}{MC} \\
PM &= MC\tan\angle BCE \\
   &= 2\times\frac{3}{4} \\
   &= \frac{3}{2}
\end{aligned}
\end{equation*}
\begin{equation*}
\begin{aligned}
\therefore\text{area of }\bigtriangleup BPC &= \frac{1}{2}BC\times PM \\
		&= \frac{1}{2}\left(4\right)\left(\frac{3}{2}\right) \\
		&= 3
\end{aligned}
\end{equation*}

\end{answer}
\end{document}